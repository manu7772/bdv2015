\subsection*{From Symfony 2.\+0 to Symfony 2.\+1 }

\subsubsection*{Project Dependencies}

As of Symfony 2.\+1, project dependencies are managed by \href{http://getcomposer.org/}{\tt Composer}\+:


\begin{DoxyItemize}
\item The {\ttfamily bin/vendors} script can be removed as {\ttfamily composer.\+phar} does all the work now (it is recommended to install it globally on your machine).
\item The {\ttfamily deps} file need to be replaced with the {\ttfamily composer.\+json} one.
\item The {\ttfamily composer.\+lock} is the equivalent of the generated {\ttfamily deps.\+lock} file and it is automatically generated by Composer.
\end{DoxyItemize}

Download the default \href{https://raw.github.com/symfony/symfony-standard/2.1/composer.json}{\tt `composer.json`} and \href{https://raw.github.com/symfony/symfony-standard/2.1/composer.lock}{\tt `composer.lock`} files for Symfony 2.\+1 and put them into the main directory of your project. If you have customized your {\ttfamily deps} file, move the added dependencies to the {\ttfamily composer.\+json} file (many bundles and P\+H\+P libraries are already available as Composer packages -- search for them on \href{http://packagist.org/}{\tt Packagist}).

Remove your current {\ttfamily vendor} directory.

Finally, run Composer\+: \begin{DoxyVerb}$ composer.phar install
\end{DoxyVerb}


Note\+: You must complete the upgrade steps below so composer can successfully generate the autoload files.

\subsubsection*{{\ttfamily app/autoload.\+php}}

The default {\ttfamily autoload.\+php} reads as follows (it has been simplified a lot as autoloading for libraries and bundles declared in your {\ttfamily composer.\+json} file is automatically managed by the Composer autoloader)\+: \begin{DoxyVerb}<?php

use Doctrine\Common\Annotations\AnnotationRegistry;

$loader = include __DIR__.'/../vendor/autoload.php';

// intl
if (!function_exists('intl_get_error_code')) {
    require_once __DIR__.'/../vendor/symfony/symfony/src/Symfony/Component/Locale/Resources/stubs/functions.php';

    $loader->add('', __DIR__.'/../vendor/symfony/symfony/src/Symfony/Component/Locale/Resources/stubs');
}

AnnotationRegistry::registerLoader(array($loader, 'loadClass'));

return $loader;
\end{DoxyVerb}


\subsubsection*{{\ttfamily app/config/config.\+yml}}

The {\ttfamily framework.\+charset} setting must be removed. If you are not using {\ttfamily U\+T\+F-\/8} for your application, override the {\ttfamily get\+Charset()} method in your {\ttfamily App\+Kernel} class instead\+: \begin{DoxyVerb}class AppKernel extends Kernel
{
    public function getCharset()
    {
        return 'ISO-8859-1';
    }

    // ...
}
\end{DoxyVerb}


You might want to add the new {\ttfamily strict\+\_\+requirements} parameter to {\ttfamily framework.\+router} (it avoids fatal errors in the production environment when a link cannot be generated)\+: \begin{DoxyVerb}framework:
    router:
        strict_requirements: %kernel.debug%
\end{DoxyVerb}


You can even disable the requirements check on production with {\ttfamily null} as you should know that the parameters for U\+R\+L generation always pass the requirements, e.\+g. by validating them beforehand. This additionally enhances performance. See \href{https://github.com/symfony/symfony-standard/blob/master/app/config/config_prod.yml}{\tt config\+\_\+prod.\+yml}.

The {\ttfamily default\+\_\+locale} parameter is now a setting of the main {\ttfamily framework} configuration (it was under the {\ttfamily framework.\+session} in 2.\+0)\+: \begin{DoxyVerb}framework:
    default_locale: %locale%
\end{DoxyVerb}


The {\ttfamily auto\+\_\+start} setting under {\ttfamily framework.\+session} must be removed as it is not used anymore (the session is now always started on-\/demand). If {\ttfamily auto\+\_\+start} was the only setting under the {\ttfamily framework.\+session} entry, don't remove it entirely, but set its value to {\ttfamily $\sim$} ({\ttfamily $\sim$} means {\ttfamily null} in Y\+A\+M\+L) instead\+: \begin{DoxyVerb}framework:
    session: ~
\end{DoxyVerb}


The {\ttfamily trust\+\_\+proxy\+\_\+headers} setting was added in the default configuration file (as it should be set to {\ttfamily true} when you install your application behind a reverse proxy)\+: \begin{DoxyVerb}framework:
    trust_proxy_headers: false
\end{DoxyVerb}


An empty {\ttfamily bundles} entry was added to the {\ttfamily assetic} configuration\+: \begin{DoxyVerb}assetic:
    bundles: []
\end{DoxyVerb}


The default {\ttfamily swiftmailer} configuration now has the {\ttfamily spool} setting configured to the {\ttfamily memory} type to defer email sending after the response is sent to the user (recommended for better end-\/user performance)\+: \begin{DoxyVerb}swiftmailer:
    spool: { type: memory }
\end{DoxyVerb}


The {\ttfamily jms\+\_\+security\+\_\+extra} configuration was moved to the {\ttfamily security.\+yml} configuration file.

\subsubsection*{{\ttfamily app/config/config\+\_\+dev.\+yml}}

An example of how to send all emails to a unique address was added\+: \begin{DoxyVerb}#swiftmailer:
#    delivery_address: me@example.com
\end{DoxyVerb}


\subsubsection*{{\ttfamily app/config/config\+\_\+test.\+yml}}

The {\ttfamily storage\+\_\+id} setting must be changed to {\ttfamily session.\+storage.\+mock\+\_\+file}\+: \begin{DoxyVerb}framework:
    session:
        storage_id: session.storage.mock_file
\end{DoxyVerb}


\subsubsection*{{\ttfamily app/config/parameters.\+ini}}

The file has been converted to a Y\+A\+M\+L file which reads as follows\+: \begin{DoxyVerb}parameters:
    database_driver:   pdo_mysql
    database_host:     localhost
    database_port:     ~
    database_name:     symfony
    database_user:     root
    database_password: ~

    mailer_transport:  smtp
    mailer_host:       localhost
    mailer_user:       ~
    mailer_password:   ~

    locale:            en
    secret:            ThisTokenIsNotSoSecretChangeIt
\end{DoxyVerb}


Note that if you convert your parameters file to Y\+A\+M\+L, you must also change its reference in {\ttfamily app/config/config.\+yml}.

\subsubsection*{{\ttfamily app/config/routing\+\_\+dev.\+yml}}

The {\ttfamily \+\_\+assetic} entry was removed\+: \begin{DoxyVerb}#_assetic:
#    resource: .
#    type:     assetic
\end{DoxyVerb}


\subsubsection*{{\ttfamily app/config/security.\+yml}}

Under {\ttfamily security.\+access\+\_\+control}, the default rule for internal routes was changed\+: \begin{DoxyVerb}security:
    access_control:
        #- { path: ^/_internal/secure, roles: IS_AUTHENTICATED_ANONYMOUSLY, ip: 127.0.0.1 }
\end{DoxyVerb}


Under {\ttfamily security.\+providers}, the {\ttfamily in\+\_\+memory} example was updated to the following\+: \begin{DoxyVerb}security:
    providers:
            in_memory:
                memory:
                    users:
                        user:  { password: userpass, roles: [ 'ROLE_USER' ] }
                        admin: { password: adminpass, roles: [ 'ROLE_ADMIN' ] }
\end{DoxyVerb}


\subsubsection*{{\ttfamily app/\+App\+Kernel.\+php}}

The following bundles have been added to the list of default registered bundles\+: \begin{DoxyVerb}new JMS\AopBundle\JMSAopBundle(),
new JMS\DiExtraBundle\JMSDiExtraBundle($this),
\end{DoxyVerb}


You must also rename the Doctrine\+Bundle from\+: \begin{DoxyVerb}new Symfony\Bundle\DoctrineBundle\DoctrineBundle(),
\end{DoxyVerb}


to\+: \begin{DoxyVerb}new Doctrine\Bundle\DoctrineBundle\DoctrineBundle(),
\end{DoxyVerb}


\subsubsection*{{\ttfamily web/app.\+php}}

The default {\ttfamily web/app.\+php} file now reads as follows\+: \begin{DoxyVerb}<?php

use Symfony\Component\ClassLoader\ApcClassLoader;
use Symfony\Component\HttpFoundation\Request;

$loader = require_once __DIR__.'/../app/bootstrap.php.cache';

// Use APC for autoloading to improve performance.
// Change 'sf2' to a unique prefix in order to prevent cache key conflicts
// with other applications also using APC.
/*
$loader = new ApcClassLoader('sf2', $loader);
$loader->register(true);
*/

require_once __DIR__.'/../app/AppKernel.php';
//require_once __DIR__.'/../app/AppCache.php';

$kernel = new AppKernel('prod', false);
$kernel->loadClassCache();
//$kernel = new AppCache($kernel);
$request = Request::createFromGlobals();
$response = $kernel->handle($request);
$response->send();
$kernel->terminate($request, $response);
\end{DoxyVerb}


\subsubsection*{{\ttfamily web/app\+\_\+dev.\+php}}

The default {\ttfamily web/app\+\_\+dev.\+php} file now reads as follows\+: \begin{DoxyVerb}<?php

use Symfony\Component\HttpFoundation\Request;

// If you don't want to setup permissions the proper way, just uncomment the following PHP line
// read http://symfony.com/doc/current/book/installation.html#configuration-and-setup for more information
//umask(0000);

// This check prevents access to debug front controllers that are deployed by accident to production servers.
// Feel free to remove this, extend it, or make something more sophisticated.
if (isset($_SERVER['HTTP_CLIENT_IP'])
    || isset($_SERVER['HTTP_X_FORWARDED_FOR'])
    || !in_array(@$_SERVER['REMOTE_ADDR'], array(
        '127.0.0.1',
        '::1',
    ))
) {
    header('HTTP/1.0 403 Forbidden');
    exit('You are not allowed to access this file. Check '.basename(__FILE__).' for more information.');
}

$loader = require_once __DIR__.'/../app/bootstrap.php.cache';
require_once __DIR__.'/../app/AppKernel.php';

$kernel = new AppKernel('dev', true);
$kernel->loadClassCache();
$request = Request::createFromGlobals();
$response = $kernel->handle($request);
$response->send();
$kernel->terminate($request, $response);\end{DoxyVerb}
 